\documentclass[11pt,a4paper]{article}

\usepackage[utf8]{inputenc}
\usepackage[T1]{fontenc}
\usepackage[spanish]{babel}
\usepackage{amsmath, amssymb}
\usepackage{listings}
\usepackage{geometry}
\usepackage{enumitem}
\geometry{margin=2cm}

\lstset{
  language=C++,
  basicstyle=\ttfamily\small,
  numbers=left,
  numberstyle=\tiny,
  stepnumber=1,
  numbersep=5pt,
  showstringspaces=false,
  breaklines=true,
  tabsize=2
}

\title{Documentación Técnica – ICPC 2025}
\author{Equipo XYZ – Universidad ABC}
\date{\today}

\begin{document}

\maketitle
\tableofcontents
\newpage

\section{Plantillas de Código}

\subsection{I/O rápido en C++}
Se encargan de desincronizar el cin y el cout con el fin de eliminar el overhead y alcanzar una velocidad similar al printf y scanf de C.
\begin{lstlisting}
ios::sync_with_stdio(false);
cin.tie(nullptr);
\end{lstlisting}

\subsection{Macros utiles para el lenguaje}
\begin{enumerate} [label=\alph*.]
  \item Atajos de sintaxis
\begin{lstlisting}
  #define pb push_back //metodo de insercion comun en vectores
  #define mp make_pair //construccion de pares en ejecucion para maps
  #define all(x) (x).begin(), (x).end() //util para sort y otros metodos que requieren iteradorres inicio y fin
  #define rall(x) (x).rbegin(), (x).rend() //igual que all pero en orden inverso
  #define sz(x) (int)((x).size()) //longitud de las principales DS std
\end{lstlisting}
  \item Tipos de datos frecuentes
\begin{lstlisting}
  #define ll long long //entero 64 bits (maximo espacio c++)
  #define ull unsigned long long //igual a ll, para casos sin negativos
  //vectores de los tipos principales
  #define vi  vector<int>
  #define vll vector<long long>
  #define vd  vector<double>
  #define vs  vector<string>
\end{lstlisting}
  \item Bucles rapidos
\begin{lstlisting}
  #define FOR(i,a,b) for(long long i=(a); i<(b); i++) //Bucle for 1 a 1 desde a hasta b
  #define REP(i,n)   for(long long i=0; i<(n); i++) //Bucle for 1 a 1 de 0 a n (Ideal para recorrer todo un vector)
  #define ROF(i,a,b) for(long long i=(a); i>=(b); i--) //Bucle descendente 1 a 1 de a hasta b
\end{lstlisting}
  \item Constantes
\begin{lstlisting}
  #define INF 1000000000 //Posible infinito para int 
  #define LINF 1000000000000000000LL //Posible infinito para long long
  #define MOD 1000000007 //Constante 10^9 + 7 para problemas gigantes
  #define MOD2 998244353 //Otra constante modular menos usada
  #define EPS 1e-9 //Margen de error para operaciones punto flotante
\end{lstlisting}
  \item Debugging rapido (No sabemos debuggear)
\begin{lstlisting}
  #define debug(x) cerr << #x << " = " << (x) << endl //Se usa cerr ya que no es evaluado por el juez
  #define debugv(v) { cerr << #v << " = "; for(auto _ : v) cerr << _ << " "; cerr << endl; }
\end{lstlisting}
\end{enumerate}

\section{Tablas de Complejidad}
\begin{itemize}
  \item \texttt{vector}: acceso $O(1)$, inserción al final $O(1)$ amortizado
  \item \texttt{set/map}: inserción/búsqueda $O(\log n)$
  \item \texttt{unordered\_map}: inserción/búsqueda $O(1)$ promedio
\end{itemize}

\section{Fórmulas Matemáticas}
\[
\sum_{i=1}^n i = \frac{n(n+1)}{2}
\]
\[
\text{Área de triángulo (Herón)} = \sqrt{s(s-a)(s-b)(s-c)}
\]

\section{Estrategia de Equipo}
Roles, protocolos de clarifications, rotación de teclado.

\end{document}